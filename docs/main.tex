% !TEX program = xelatex
\documentclass[a4paper, 12pt]{report}

\usepackage{fontspec}
\usepackage[romanian]{babel}
\usepackage{graphicx}

\usepackage[backend=biber]{biblatex}
\DeclareLanguageMapping{romanian}{romanian}
\addbibresource{bibliography.bib}

\usepackage{setspace}
\onehalfspace

\usepackage{csquotes}
\DeclareQuoteStyle{romanian}
  {\quotedblbase}
  {\textquotedblright}
  {\guillemotleft}
  {\guillemotright}

\usepackage{listings, listings-rust}

\title{Titlu}
\author{Alexandru-Gabriel Ică}
\date{\today}
\makeatletter

\begin{document}

\begin{titlepage}
    \begin{figure}[!htb]
    \centering
    \begin{minipage}{0.19\textwidth}
        \includegraphics[width=\linewidth]{img/UB_Logo.png}
    \end{minipage}
    \begin{minipage}{0.57\textwidth}
        \large
        \vspace{0.2cm}
        \begin{center}
            \textbf{Universitatea din București}
        \end{center}
        \vspace{0.3cm}
        \begin{center}
            \textbf{
                Facultatea de \\
                Matematică și Informatică
            }
        \end{center}
    \end{minipage}
    \begin{minipage}{0.21\textwidth}
        \includegraphics[width=\linewidth]{img/FMI_Logo.png}
    \end{minipage}
    \end{figure}

    \begin{center}
        Specializarea Informatică
    \end{center}

    \vspace{0.5cm}

    \begin{center}
        {\Large Lucrare de licență}
    \end{center}

    \begin{center}
        {\huge \@title}
    \end{center}

    \vspace{2.8cm}

    \begin{center}
        \large Absolvent \\ \@author
    \end{center}

    \vspace{0.25cm}

    \begin{center}
        \large Coordonator științific \\ Prof. Stupariu-Mihai Sorin
    \end{center}

    \vspace{2cm}

    \begin{center}
        \Large București, Iunie 2022
    \end{center}
\end{titlepage}
\makeatother

\tableofcontents

\chapter{Motivație}

Grafica vectorială \cite{vector_graphics_wikipedia} reprezintă un mod de a genera imagini
direct din elemente geometrice precum linii drepte sau linii curbe. Acest mod de a reprezenta
imaginile diferă de modul „standard”, adică grafica rasterizată, prin faptul că randarea imaginii
nu își pierde calitatea indiferent de rezoluția la care vrem să o afișăm. De asemenea, imaginile
pot fi stocate mai eficient din punct de vedere al spațiului deoarece nu trebuie să reținem pixeli
la rezoluții mari.

Grafica vectorială este folosită foarte des, una din principalele utilizări fiind randarea textului
pe ecran. Se poate regăsi și în interfețe grafice (de exemplu iconițe sau butoane) sau pe site-uri
web (în format SVG \cite{svg_standard}).

În ciuda faptului că grafica vectorială este des întâlnită, opțiunile pe care le avem pentru a putea
integra pe aceasta în proiecte nu sunt chiar triviale, menționăm câteva din cele mai folosite biblioteci:
\begin{itemize}
    \item{Skia \cite{skia_library}: o bibliotecă puternică, cu multe funcționalități, dar care suferă
        de un proces de integrare destul de complex, aceasta depinzând de Bazel \cite{bazel_build_system},
        care, la rândul lui, depinde de Java \cite{java_programming_language}}
    \item{Cairo \cite{cairo_library}: din nou o bibliotecă cu multe funcționalități, dar care suferă de aceeași
        problemă legată de procesul de integrare, mai ales pe sistemele de operare care nu sunt tip UNIX \cite{unix}}
    \item{Bibliotecile care vin la pachet cu diverse sisteme de operare: acestea sunt avantajoase deoarece pot fi
        integrate ușor. Inconvenientul este că sunt diferențe destul de mari între implementări, diferențe și de
        performanță și de calitate, este destul de dificil să se obțină un rezultat identic pe platforme diferite}
\end{itemize}

\section{Problema abordată}

Ceea ce vreau să creez este o bibliotecă de grafică vectorială care să fie ușor de integrat într-un proiect,
care să fie de o calitate decentă și care să aibă aceeași calitate pe platforme diferite.
Pentru acest fapt am ales să folosesc limbajul Rust \cite{rust_lang}. Motivul principal este că Rust are un manager
de pachete standard, ușor de folosit. Alt motiv este faptul că Rust este printre singurele limbaje populare care să
dispună de un astfel de manager de pachete, și care să concureze în aceeași nișă ca limbajele în care sunt implementate
bibliotecile menționate mai sus (C++ și C). Acest manager de pachete face integrarea într-un proiect mult mai ușoară.

\section{Obiective}

Obiectivul principal al acestei lucrări este o bibliotecă de grafică vectorială simplă, ușor de folosit și de integrat
într-un proiect, de o calitate decentă, cu funcționalități standard, care să aibă aceeași calitate pe orice platformă
suportată.

\section{Alte proiecte relevante}

\chapter{Algoritmul de bază}

\section{Algoritmul de la baza AntiGrain}

\section{Algoritmul de intersecție între linii}

\section{Aplicarea algoritmului pe un set de date}

\section{Cum tratăm liniile verticale?}

\chapter{Extinderea algoritmului la mai multe primitive}

\section{Curbe Bézier}

\section{Transformarea curbelor Bézier pătratice în set de linii}

\section{Transformarea curbelor Bézier cubice în set de linii}

\chapter{Efecte}

\section{Gradienți}

\subsection{Liniari}
\subsection{Radiali}
\subsection{Conici}

\section{Amestecare}
\subsection{Amestecarea mai multor figuri făcând parte din același grup}
\subsection{Porter-Duff}

\section{Decuparea}

\section{Mascarea}

\section{Spații de culoare}

\chapter{Exemplu de cod Rust}

\begin{figure}[ht]
    \centering
    \begin{lstlisting}[language=Rust]
    fn main() {
        println!("Hello {}!", "world");
    }
    \end{lstlisting}
    \caption{Un simplu 'Hello World!'}
    \label{fig-rust}
\end{figure}

\printbibliography

\end{document}